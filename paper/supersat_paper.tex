\documentclass{article}

\usepackage[utf8]{inputenc}
\usepackage{graphicx}
\usepackage{url}
\usepackage{color}
\usepackage{titlesec}
\usepackage{amsmath}
\usepackage{physics}
\usepackage{amsfonts}
\usepackage{subcaption}
\usepackage{booktabs}
\usepackage[counterclockwise]{rotating}
\graphicspath{{../figures/}}

\title{Paper draft v1.3 for supersat project}
\author{K. Latimer}
\date{Jan 11, 2020}

\newcommand{\drcomm}[1]{\textcolor{blue}{\textit{#1}}}
\newcommand{\klcomm}[1]{\textcolor{red}{\textit{#1}}}
\newcommand{\todo}[1]{\textcolor{green}{\textit{#1}}}

\begin{document}

\maketitle

\noindent\drcomm{Questions/comments from DR in blue} \\
\noindent\klcomm{Responses from KL in red}\\
\noindent\todo{Hanging details to address before final draft}\\

\section{Intro}

In a recent paper, Fan et al introduce a novel ``warm phase invigoration mechanism" (WPIM) in which increased concentrations of ultrafine aerosol particles (UAP$_{<50}$, with 50 signifying an upper bound on particle diameter of 50nm) in the boundary layer (BL) result in enhanced convective updraft speeds and precipitation rates \cite{Fan2018}. As pointed out by Grabowski and Morrison, the precise explanation for this physical effect is that, in allowing for lower equilibrium water vapor supersaturation (SS) values in rising convective parcels, these excess aerosol particles lead to an increase in the buoyancy of the parcel over the course of its ascent, thus enhancing convective speeds \cite{Grabowski2020}.

In order to get a quantitative intuition for how this works, we offer a simplified version of the calculation in \cite{Grabowski2015}, which still conveys the same essential idea. We consider a polluted (non-supersaturated; i.e. $RH=1$) storm ascending in an environment whose temperature profile has been set by clean storms. The parcel condenses water vapor as it rises, and for simplicity we assume no latent heat is lost to the environment. We then have (see Table \ref{vartable} for explanation of constants and variables used in the text. We use $\delta$ here to represent a variation in state variables between two parcels, as distinguised from $d$ in Equation \ref{dCAPE} which denotes a proper differential form):
\begin{equation}
\label{energyconsv}
C_{ap}\delta T + L_v\delta q_v = 0,
\end{equation}
where $q_v$ is the water vapor mass fraction of the parcel ($q_v=m_v/m_{tot}$), also expressed in terms of the the relative humidity ($RH$) and saturation water vapor mass fraction ($q_v^*$) as:
\begin{equation}
\label{qveqn}
q_v = RHq_v^*
\end{equation}
Usig the Clausius-Clayperon equation:
\begin{align}
\label{clauclay}
\delta q_v^* &= \delta \Big(\frac{e_sV}{R_vTm_{tot}}\Big)\nonumber\\ 
&=\frac{\delta e_s}{e_s}q_v^* - \frac{\delta T}{T}q_v^*\nonumber\\ 
&=\frac{L_v\delta T}{R_vT^2}q_v^* - \frac{\delta T}{T}q_v^*\nonumber\\ 
&=\Big(\frac{L_v}{R_vT} - 1\Big)\frac{\delta T}{T}q_v^*\nonumber\\ 
&\approx \frac{q_v^*L_v}{R_vT^2}\delta T 
\end{align}
Taking the differential of Equation \ref{qveqn} and rearranging terms in Equations \ref{energyconsv}, \ref{qveqn}, and \ref{clauclay} yields:
\begin{equation}
\label{dT}
\delta T = \frac{-L_vq_v^*}{C_{pa} + q_v\frac{L_v^2}{R_vT^2}}\delta RH
\end{equation}
Plugging in typical values for $RH$ ($\approx 1.1$) and $T$ ($\approx 300$ K) gives $dT\approx 1$ K.

In their paper (see for example Figure 2(b) of that work), Fan et al provide anecdotal evidence that the WPIM is capable of producing enhancements in vertical wind velocity on the order of 10 m/s for polluted relative to unpolluted storms. Even neglecting diffusive, frictional, or radiative losses, this requires a variation in convective available potential energy ($CAPE$) of $\approx 100$ J/kg, or a $RH$ difference between the dirty storm and clean environment of $\approx 0.1$, i.e., the environment must support SS on the order of 10\% throughout the troposphere.

While Fan et al do not offer any proof based on experimental data that such high SS exist in the convection setting the environmental lapse rate, they do observe comparable values in numerical simulations. In particular, using the Weather Research and Forcasting (WRF) model to simulate pristine (no UAP$_{<50}$) conditions in the Amazon Rainforest, they find (horizontally- and time-averaged) SS in convective cores of up to 15\%.

Since this is well above what is typically reported or assumed in the literature \cite{Hoppel1996, Yang2019, Koike2012, Politovich1988, Moteki2019, Siebert2017, Shen2018, Hammer2014, Li2019}, we seek in this paper to determine if we can find experimental evidence for O(10\%) SS in nature. We use data from the High-Altitude LOng-range research aircraft (HALO) (part of the ACRIDICON-CHUVA mission in and around Manaus, Brazil in 2014-15), as well as from the first phase of the Cloud Aerosol Interaction and Precipitation Enhancement EXperiment (CAIPEEX) in India (taken in June [around Hyderabad] and August [around Bareilly] 2009) \cite{???, Kulkarni2012} as a dataset of opportunity.

\section{Data Analysis and Results}

In order to determine experimental supersaturation within a reasonable margin of error, we must use the quasi-steady-state supersaturation ($SS_{QSS}$) formula \cite{Rogers1989}. In order to determine the region of validity for this approximation, we used data from the WRF output, since this allows us to compare to the true supersaturation (here called $SS_{WRF}$). We focused on the following considerations:
\begin{itemize}
\item The QSS approximation should be a good and reliable estimate for the actual SS, i.e. least-squares linear regression slope and correlation coefficients $\approx$1. Figure \ref{regresheatmap} shows the values of these statistics over a range of minimum liquid water content (LWC) and vertical wind velocity ($w$) cutoff values. We find that a LWC cutoff range $10^{-4}-10^{-3.5}$ g/g yields the best results overall in this respect, regardless of $w$ cutoff.
\item Assuming the QSS approximation is valid as described above, the filter should have a minimal distorting effect on the form of the SS distribution. In particular, we want to avoid truncating the high-SS tail since these are of central importance in the consideration of the WPIM. In Figure \ref{ssdistbs} we show that, given the results from the previous point, a slightly less restrictive LWC cutoff is more appropriate in this second aim. [\drcomm{The PDFs have not converged so how do we know this is OK? What high SS values are we missing? Why is it OK to ignore them?} \klcomm{Discuss Figure \ref{ssdistbswithfull} in person.}]
\end{itemize}
 We therefore settled on the following criteria, which are used in all subsequent analyses:
\begin{itemize}
	\item T \textgreater  273K (we're not including ice in the theory; note that Fan et al do evaluate SS wrt water above the freezing line though)
	\item w \textgreater  2 m/s (reasonably strong updrafts) [\todo{w cutoff 1 vs 2 m/s}]
	\item cloud LWC \textgreater  1e-4 g/g (in the convection core) [\drcomm{If (logarithmic vertical grid) is the issue, then you could just plot the area fraction as a function of height. Does the sudden dropoff with height go away when that is plotted?} \klcomm{That is equivalent to what I did in Figures \ref{wrfbipanelv14} and \ref{wrfbipanelv16}, up to a scaling factor. The dropoff is more gradual but still there. But it doesn't seem like the LWC cutoff is the critical aspect here because Figure \ref{wrfbipanelv16} has no such cutoff and yet doesn't look drastically different from Figure \ref{wrfbipanelv14}.}]
	\item including rain droplets and ventillation corrections
\end{itemize}

\begin{figure}[ht]
	\centering
	\begin{subfigure}{1.5\textwidth}
		\includegraphics[width=\textwidth]{revmywrf/v7_from_data_regres_param_heatmaps_Unpolluted_figure.png}
		\caption{Unpolluted case.}
		\label{regresheatmapunpoll}
	\end{subfigure}
	\begin{subfigure}{1.5\textwidth}
		\includegraphics[width=\textwidth]{revmywrf/v7_from_data_regres_param_heatmaps_Polluted_figure.png}
		\caption{Polluted case.}
		\label{regresheatmappoll}
	\end{subfigure}
	\caption{Variation of (left) least-squares slope parameter and (center) correlation coefficient for $SS_{QSS}$ vs $SS_{WRF}$ with respect to cutoff values for LWC and w (in addition to imposing temperature cutoff, and inclusion of raindrops and ventilation corrections). Right panel shows Euclidean distance from the ideal point (1, 1) in the $m-R^2$ plane. [\todo{Fix formatting; metric for rightmost panel?}]}
	\label{regresheatmap}
\end{figure}
\begin{figure}[ht]
	\centering
	\begin{subfigure}{1\textwidth}
		\includegraphics[width=\textwidth]{revmywrf/v7_from_data_ss_distb_charts_Unpolluted_figure.png}
		\caption{Unpolluted case.}
		\label{ssdistbsunpoll}
	\end{subfigure}
	\begin{subfigure}{1\textwidth}
		\includegraphics[width=\textwidth]{revmywrf/v7_from_data_ss_distb_charts_Polluted_figure.png}
		\caption{Polluted case.}
		\label{ssdistbspoll}
	\end{subfigure}
	\caption{Normalized SS distributions after imposing filtering criteria on LWC and w (in addition to temperature cutoff, and inclusion of raindrops and ventilation corrections).}
	\label{ssdistbs}
\end{figure}
\begin{figure}[ht]
	\centering
	\begin{subfigure}{1\textwidth}
		\includegraphics[width=\textwidth]{revmywrf/v8_from_data_ss_distb_charts_Unpolluted_figure.png}
		\caption{Unpolluted case.}
		\label{ssdistbsunpollwithfull}
	\end{subfigure}
	\begin{subfigure}{1\textwidth}
		\includegraphics[width=\textwidth]{revmywrf/v8_from_data_ss_distb_charts_Polluted_figure.png}
		\caption{Polluted case.}
		\label{ssdistbspollwithfull}
	\end{subfigure}
	\caption{Same as above but with more values of cutoff parameters [\klcomm{not for final draft}]}
	\label{ssdistbswithfull}
\end{figure}

Figure \ref{wrfvsqss} shows a scatterplot with the agreement between the actual and QSS-derived SS values in the WRF simulation, for points satisfying the above criteria. For brevity we will henceforth refer to such points as `cloudy updrafts.'

\clearpage
\newpage

\begin{figure}[ht]
	\centering
	\begin{subfigure}{0.7\textwidth}
		\includegraphics[width=\textwidth]{revmywrf/v9_FINAL_heatmap_ss_qss_vs_ss_wrf_Unpolluted_figure.png}
		\caption{Unpolluted case.}
		\label{wrfvsqssunpoll}
	\end{subfigure}
	\begin{subfigure}{0.7\textwidth}
		\includegraphics[width=\textwidth]{revmywrf/v9_FINAL_heatmap_ss_qss_vs_ss_wrf_Polluted_figure.png}
		\caption{Polluted case.}
		\label{wrfvsqsspoll}
	\end{subfigure}
	\caption{Actual ($SS_{WRF}$) vs predicted ($SS_{QSS}$) supersaturation. Color indicates density of data points; note the scale is logarithmic.}
	\label{wrfvsqss}
\end{figure}

In their analysis, Fan et al use the following filtering criteria:
\begin{itemize}
	\item For experimental data: They examine the upper 10th percentile of updrafts in ``convective events" which 1) Fall between 11h00 and 19h00 local time 2) Have no other convective events occuring at any point in time 3 hours prior, and 3) Have max echo height at $>$0dBz above 10km. For aerosol measurements they take the average of measurements in the 30-min interval prior to the convective event. The authors do not state their criteria for what qualifies as a convective event.
	\item For model data: They limit analysis to a subset of the horizontal domain (red box in Fig S8) that encompasses a single convective event during the day of the simulation. They again take the top 10th percentile of updrafts with $w>2$. 
\end{itemize}

Because we use slightly different filtering criteria than that described above, we first verify that our selected cloudy updraft points yield similar SS profiles in the WRF simulation as those found in the latter work. In Figure \ref{wrfbipanel} we plot vertical SS profiles for all cloudy updraft points, as well as for the upper 10 percentiles of all cloudy updraft points. We define the points in the ``upper 10 percentiles" as those with $w$ greater than the 90th-percentile (out of all altitudes and times) vertical wind velocity. Data in Figure \ref{wrfbipanel} are binned according to the simulation grid, which is based on pressure coordinates so that bin size varies logarithmically with respect to $z$. For this and future vertical profiles, the vertical coordinate of plotted points represents the average over all vertical coordinates of points in the corresponding bins. In \cite{Fan2018} they consider a restricted subdomain around the T3 field station, indicated in Figure S8 of the supplementary information for that paper; we do not see a major qualitative difference when including this additional criterion; see Figure \ref{wrfsubdombipanel}. We do indeed find the high SS values reported by Fan et al (maximum values of 13\% in both polluted and unpolluted cases from the upper 10th percentile dataset), confirming that our filtering criteria establish a fair basis for comparison here.

\begin{figure}[ht]
	\centering
	\begin{subfigure}{0.7\textwidth}
		\includegraphics[width=\textwidth]{revmywrf/v5_FINAL_bipanel_ss_qss_vs_z_allpts_figure.png}
		\caption{}
		\label{wrfbipanelallpts}
	\end{subfigure}
	\begin{subfigure}{0.7\textwidth}
		\includegraphics[width=\textwidth]{revmywrf/v5_FINAL_bipanel_ss_qss_vs_z_up10perc_figure.png}
		\caption{}
		\label{wrfbipanelup50perc}
	\end{subfigure}
	\caption{SS profiles (left) and probability density of sampled points (right) for a) all cloudy updraft points; b) top 10\% of cloudy updraft points with respect to vertical wind velocity, in WRF simulation conducted by Fan et al. The coordinates for points plotted in a) represent the average (of SS and $z$ values) over all points in the corresponding vertical bin. The vertical bin interval varies logarithmically with respect to $z$.}
	\label{wrfbipanel}
\end{figure}

\clearpage
\newpage

We now seek to determine whether such high values actually occur in nature. First we look at data from the HALO flights in September and October of 2014 (see Methods/SI for details on selection of dates for this analysis) [\todo{find citation}]. Figure \ref{halobipanel} shows the analogue of Figure \ref{wrfbipanel} using data from all HALO flight dates combined. We find no points with average SS above 1 \%, even when limiting to the strongest updrafts in the combined dataset.

\begin{figure}[ht]
    \centering
    \includegraphics[width=9cm]{revhalo/v4_FINAL_combined_bipanel_ss_qss_vs_z_figure.png}
    \caption{SS profiles (left) and probability density of sampled points (right) from HALO flight campaign (all dates combined). [\drcomm{Why do the top 10\% w values occur at only four heights?} \klcomm{Because I compare to 90th-percentile $w$ out of all data points. Most altitude bins have less than ten points so it doesn't make sense to take upper 10 percentiles from each altitude bin.} \drcomm{I am confused. If you are using the 90th percentile of all cloudy updrafts (incidentally, state what value that is), why would it matter if a given altitude bin has 10 points?} \klcomm{Because If there are less than ten points in the bin then there is less than one point in the upper ten percentiles of updrafts.} \drcomm{By definition, something's number density would give the number of that something when integrated.  Something's probability density would give 1 when integrated.} \klcomm{Noted. I changed wording to `probability density' where appropriate.} \todo{If there is a reason to indicate number rather than probability density I can edit the figures}] SS profile is plotted with markers so as not to obscure intervals with missing data. The coordinates for points plotted in a) represent the average (of SS and $z$ values) over all points in the corresponding vertical bin. The vertical bin width is constant with respect to $z$.}
    \label{halobipanel}
\end{figure}

Finally, we examine a second experimental dataset from the first phase of the CAIPEEX campaign \cite{Kulkarni2012}. Although no UAP$_{<50}$ concentration measurements are available during this phase of the experiment, measurements of aerosols with diameters in the range of 0.1-3 $\mu$m showed total aerosol concentrations ranging from 700/ccm to 2500/ccm in the BL (see Figure 3(b) in \cite{Prabha2011} and Figure 4(a) in \cite{Kulkarni2012}). Reliable rain drop particle size distributions are unavailable from the flight dates in this analysis phase of the experiment, but we observe that exclusion of raindrops from the calculation of QSS SS leads to a systematic overestimation of the true SS (see Methods/SI). Therefore we take the SS profiles in Figure \ref{caipeexbipanel} as an upper bound. We observe slightly higher values relative to those from the HALO flights, although we still don't find such high values as those output by the WRF models for the middle troposphere.


\begin{figure}[ht]
    \centering
    \includegraphics[width=9cm]{revcaipeex/v4_FINAL_combined_bipanel_ss_qss_vs_z_figure.png}
    \caption{SS profiles (left) and probability density of sampled points (right) from CAIPEEX flight campaign (all dates combined). SS profile is plotted with markers so as not to obscure intervals with missing data. The coordinates for points plotted in a) represent the average (of SS and $z$ values) over all points in the corresponding vertical bin. The vertical bin interval is constant with respect to with $z$. See text for comments on the dataset.}
    \label{caipeexbipanel}
\end{figure}

We can use Equation \ref{dT} and the SS profiles in Figures \ref{wrfbipanel}-\ref{caipeexbipanel} to infer a buoyancy profile for a hypothetical non-supersaturated parcel. For this analysis we take the temperature of the parcel equal to that of the environment (i.e., what has been measured). The resulting error in the value of $\delta T$ in Equation \ref{dCAPE} is quadratic in $\delta RH$, which is acceptable for our purposes. In Figure \ref{dTprofiles} we plot $\delta T$ profiles from both WRF simulations and field campaigns side-by-side. We use these profiles to derive enhancements in $CAPE$ for the non-supersaturated parcel as:
\begin{equation}
\label{dCAPE}
\delta CAPE = g \int dz \frac{\delta T}{T}
\end{equation}
where we again approximate $T$ as the environmental temperature and integrate from 647 to 4488 m, the common vertical domain for all four curves in Figure \ref{dTprofiles}. We find values of $\delta CAPE$ of 4, 6, 36, and 68 J/kg for HALO, CAIPEEX, WRF (polluted), and WRF (unpolluted), respectively. Neglecting any other physical energy sinks as above, these translate to vertical velocity enhancements of about 3 m/s in the field campaigns and 10 m/s in the simulations.

\begin{figure}[ht]
    \centering
    \includegraphics[width=12cm]{revmywrf/v1_FINAL_combined_dT_profile_figure.png}
    \caption{Profiles for $\delta T$ of a non-supersaturated ($RH=1$) parcel ascending in an environment with SS profiles shown in Figures \ref{wrfbipanel}-\ref{caipeexbipanel}, using Equation \ref{dCAPE}. SS profiles for HALO and CAIPEEX are plotted with markers so as not to obscure intervals with missing data.}
    \label{dTprofiles}
\end{figure}

One possible counterargument is that the aerosol concentrations in the BL during the dates of the HALO flights might have been significantly higher than those during the dates considered in Fan's paper, thus precluding the occurence of high SS values in the troposphere. In order to investigate this, we use the aerosol particle size distribution measured by the scanning mobility particle sizer (SMPS) in Manacapuru, located southwest of Manaus (PI: Chongai Kuang). This intrument measures particle concentrations in the diameter range 11.1-469.8nm. In Figure \ref{goamahist}, we show that, while we do indeed see higher total aerosol concentrations on average during the HALO flight date range (3500/ccm vs 2400/ccm), the UAP50 concentration is on average lower (670/ccm vs 1600/ccm). In fact, the aerosol concentrations used in the WRF simulations are much lower than those observed during the day the simulation takes place, which is not justified quantitatively in that study.

We note additionally that the positive experimental correlation between concentration of UAP50 and maximum vertical velocity during the dates studied by Fan et al is not significant at the 95\% confidence level - the least-squares slope parameter for their data set (the plot of which we reproduce in Figure \ref{fans2a}, with additional 95\% confidence bands) has a p-value of 0.11.
\begin{figure}[ht]
	\centering
	\begin{subfigure}{0.7\textwidth}
		\includegraphics[width=\textwidth]{goama/v1_FINAL_tot_compare_nconc_hist_alldates_figure.png}
		\label{goamatothist}
		\caption{}
	\end{subfigure}
	\begin{subfigure}{0.7\textwidth}
		\includegraphics[width=\textwidth]{goama/v1_FINAL_uap50_compare_nconc_hist_alldates_figure.png}
		\label{goamauap50hist}
		\caption{}
	\end{subfigure}
	\caption{Distribution of aerosol concentration measurements by the ground-based SMPS at Manacapuru, Brazil; a) entire size range, b) only particles with diameter greater than 50nm. HALO flight dates are the same as those represented in Figure \ref{halobipanel} (see Methods/SI for details). Dashed (dotted) lines show initial concentrations in the BL of the WRF simulation of polluted (unpolluted) conditions.}
	\label{goamahist}
\end{figure}
\begin{figure}[ht]
    \centering
    \includegraphics[width=12cm]{revhalo/v2_FINAL_fan_fig_s2a.png}
    \caption{Ground-based total (including UAP50) aerosol concentration measurements versus maximum vertical wind velocity in convective cores; reproduced from Figure S2(a) of \cite{Fan2018} with additional confidence bands our own.}
    \label{fans2a}
\end{figure}

Another possible counterargument is that the flight campaigns simply didn't fly through strong enough updrafts. However the vertical velocity distributions from the campaigns are quite similar to that from the simulations. See Figure \ref{combinedwhist}. 

\begin{figure}[ht]
    \centering
    \includegraphics[width=12cm]{revmywrf/v2_FINAL_combined_w_hist_figure.png}
    \caption{Vertical wind velocity distribution from simulations and field campaigns. Using filtering criteria outlined in the text.}
    \label{combinedwhist}
\end{figure}

Conclusion: The WPIM as proposed by Fan et al requires the average temperature profile of the troposphere to be set by relatively clean (high-SS) convection, in order for more polluted (low-SS) convection to experience an enhancement in buoyancy. However, we find no evidence that the high SS values reported by Fan's model actually occur in nature, which weakens the possibility of measureable invigoration effects - in particular, we estimate an upper bound on vertical velocity enhancement of $\approx$ 3 m/s from the HALO and CAIPEEX flight campaigns, compared to $\approx$ 10 m/s from Fan's control simulations in WRF. The relatively low aerosol concentrations used to initialize the simulations, in combination with possible irregularities in microphysical parameterizations, may be to blame for the anomalously high SS values in the WRF output.

\clearpage
\newpage

\section{Methods/SI}

\subsection{WRF}

Model output for control simulations of polluted (``C\_BG") and unpolluted (``C\_PI") scenarios were provided by Fan et al; see that paper and accompanying SI for detailed explanations of model parameters and initializations.

In this paper, we use the following form of the QSS SS equation after \cite{Rogers1989} (with $SS_{QSS}$ given as a percentage):
\begin{equation}
\label{fullss}
SS_{QSS} = \frac{A(T) w}{4\pi B(\rho_a, T) \langle f(r)\cdot r\rangle n}*100,
\end{equation}
where:
\begin{align}
A(T) &= \frac{g}{R_a T}\Big(\frac{L_v R_a}{C_{ap} R_v T} - 1\Big)\big(F_d(T) + F_k(T)\big)\nonumber\\
F_d(T) &= \frac{\rho_w R_v T}{D e_s(T)}\nonumber\\
F_k(T) &= \Big(\frac{L_v}{R_v T} - 1\Big)\frac{L_v \rho_w}{K T}\nonumber\\
B(\rho_a, T) &= \rho_w\Big(\frac{R_v T}{e_s(T)} + \frac{L_v^2}{R_v C_{ap} \rho_a T^2}\Big)
\end{align}
Notation for constants and variables is given in Table \ref{vartable}. We use the following parameterization for $e_s$ \cite{Rogers1989}:
\begin{equation}
e_s(T) = 611.2e^{\frac{17.67T_c}{T_c + 243.5}},
\end{equation}
where $T_c$ is the temperature in degrees Celsius.

We note that this equation by also include finite size correction terms; however, using typical values for droplet salinity and condensation nucleus radius (the relevant parameters in this case), these terms are insignificant ($<$ 0.1\% correction to SS) for drops of radius greater than 3 $\mu$m \cite{Rogers1989}, and we therefore do not consider them in this paper.

A simpler form of Equation \ref{fullss} is often employed in the literature \cite{Grabowski2020, Rogers1989}, with:
\begin{align}
A(T) &= \frac{g}{R_a T}\Big(\frac{L_v R_a}{C_{ap} R_v T} - 1\Big)\nonumber\\
B(T) &= D
\end{align}
Figure \ref{wrfvsqssv2} shows that this form does not yield as good of agreement with the actual SS reported in WRF.

\begin{figure}[ht]
	\centering
	\begin{subfigure}{0.7\textwidth}
		\includegraphics[width=\textwidth]{revmywrf/v2_FINAL_heatmap_ss_qss_vs_ss_wrf_Unpolluted_figure.png}
		\caption{Unpolluted case.}
		\label{wrfvsqssunpollv2}
	\end{subfigure}
	\begin{subfigure}{0.7\textwidth}
		\includegraphics[width=\textwidth]{revmywrf/v2_FINAL_heatmap_ss_qss_vs_ss_wrf_Polluted_figure.png}
		\caption{Polluted case.}
		\label{wrfvsqsspollv2}
	\end{subfigure}
	\caption{Actual ($SS_{WRF}$) vs predicted ($SS_{QSS}$) supersaturation, using simplified form of Equation \ref{fullss}. Color indicates probability density of data points; note the scale is logarithmic. [\todo{if this goes in final version, change the color scheme}]}
	\label{wrfvsqssv2}
\end{figure}

We use the expressions given in \cite{Pruppacher2010} and \cite{Rogers1989} for ventilation corrections, which are quite extensive. Full formulae are contained in the analysis code at \href{https://github.com/kt-latimer/20supersat}{this GitHub repo} [\comm{will need to clean this up}]. Figures \ref{wrfvsqssv3} and \ref{wrfvsqssv5} show, respectively, the effects of neglecting these corrections (i.e. setting $f(r)=1$ for all $r$) for rain drops (defined here as liquid water drops with diameter greater than 50 $\mu$m), and omitting rain drops altogether from the calculations of mean radius and number concentration.

\begin{figure}[ht]
	\centering
	\begin{subfigure}{0.7\textwidth}
		\includegraphics[width=\textwidth]{revmywrf/v3_FINAL_heatmap_ss_qss_vs_ss_wrf_Unpolluted_figure.png}
		\caption{Unpolluted case.}
		\label{wrfvsqssunpollv3}
	\end{subfigure}
	\begin{subfigure}{0.7\textwidth}
		\includegraphics[width=\textwidth]{revmywrf/v3_FINAL_heatmap_ss_qss_vs_ss_wrf_Polluted_figure.png}
		\caption{Polluted case.}
		\label{wrfvsqsspollv3}
	\end{subfigure}
	\caption{Actual ($SS_{WRF}$) vs predicted ($SS_{QSS}$) supersaturation, without ventilation corrections. Color indicates density of data points; note the scale is logarithmic.}
	\label{wrfvsqssv3}
\end{figure}

\begin{figure}[ht]
	\centering
	\begin{subfigure}{0.7\textwidth}
		\includegraphics[width=\textwidth]{revmywrf/v5_FINAL_heatmap_ss_qss_vs_ss_wrf_Unpolluted_figure.png}
		\caption{Unpolluted case.}
		\label{wrfvsqssunpollv5}
	\end{subfigure}
	\begin{subfigure}{0.7\textwidth}
		\includegraphics[width=\textwidth]{revmywrf/v5_FINAL_heatmap_ss_qss_vs_ss_wrf_Polluted_figure.png}
		\caption{Polluted case.}
		\label{wrfvsqsspollv5}
	\end{subfigure}
	\caption{Actual ($SS_{WRF}$) vs predicted ($SS_{QSS}$) supersaturation, without contributions from rain drops. Color indicates density of data points; note the scale is logarithmic.}
	\label{wrfvsqssv5}
\end{figure}

Finally, we note that we have excluded contributions to mean radius and number concentraiton from cloud droplets of diameter less than 5 $\mu$m, for consistency with our analysis of HALO data (see proceeding subsection). In the end this yields values for $SS_{QSS}$ which are indistinguishable to a reasonable number of significant figures.

Figures \ref{wrfbipanelv14} and \ref{wrfbipanelv15} show point distributions as absolute count rather than density with respect to $z$ (since the WRF grid is spaced linearly in pressure rather than real-space coordinates). We see that the distribution is not very sensitive to $LWC$ cutoff (which is lower by an order of magnitude in Figure \ref{wrfbipanelv15}).

\begin{figure}[ht]
	\centering
	\begin{subfigure}{0.7\textwidth}
		\includegraphics[width=\textwidth]{revmywrf/v14_FINAL_bipanel_ss_qss_vs_z_allpts_figure.png}
		\caption{}
		\label{wrfbipanelallptsv14}
	\end{subfigure}
	\begin{subfigure}{0.7\textwidth}
		\includegraphics[width=\textwidth]{revmywrf/v14_FINAL_bipanel_ss_qss_vs_z_up10perc_figure.png}
		\caption{}
		\label{wrfbipanelup50percv14}
	\end{subfigure}
	\caption{Identical to Figure \ref{wrfbipanel} but righthand panel just shows absolute number of points per $z$-interval. [\klcomm{axis label on the right should be $N_{points}$ (absolute point count) not $n_{points}$ (probability density)} \todo{remake figure if included in final version}]}
	\label{wrfbipanelv14}
\end{figure}
\begin{figure}[ht]
	\centering
	\begin{subfigure}{0.7\textwidth}
		\includegraphics[width=\textwidth]{revmywrf/v16_FINAL_bipanel_ss_qss_vs_z_allpts_figure.png}
		\caption{}
		\label{wrfbipanelallptsv16}
	\end{subfigure}
	\begin{subfigure}{0.7\textwidth}
		\includegraphics[width=\textwidth]{revmywrf/v16_FINAL_bipanel_ss_qss_vs_z_up10perc_figure.png}
		\caption{}
		\label{wrfbipanelup50percv16}
	\end{subfigure}
	\caption{Identical to Figure \ref{wrfbipanelv14} but with no $LWC$ cutoff.}
	\label{wrfbipanelv16}
\end{figure}
\clearpage
\newpage

\subsection{HALO}

The HALO aircraft supported two instruments for measuring cloud droplet spectra: a cloud and aerosol spectrometer (CAS-DPOL) and a cloud droplet probe (one element of a cloud combination probe) (CCP-CDP) \cite{Braga2017}. We found that the CCP-CDP consistently reported unphysical bimodal size distributions, and therefore used only data from the CAS-DPOL for all calculations involving cloud droplets. Number concentrations from the CAS-DPOL were corrected using the $\xi$ factor derived in \cite{Weigel2016}.

The rain drop spectra came from data collected by greyscale cloud imaging probe (second element of the cloud combination probe) (CCP-CIP). The drop diameter detection ranges for CAS-DPOL and CCP-CIP were 0.89-50 $\mu$m and 25-2000 $\mu$m, respectively. Per guidance from the principal investigators for the CAS-DPOL, we only included data for droplets from size bins with a lower diameter bound greater than 3 $\mu$m in the analysis \cite{Jurkat2020}. Effectively (given size bins for this instrument), this meant that the lower bound on diameter for water drops was 5 $\mu$m. Because the CAS-DPOL and CCP-CIP have overlapping diameter detection ranges, we use concentrations for particles between 5 and 25 $\mu$m from CAS-DPOL and from 25 to 2000 $\mu$m from CCP-CIP. 

All measurements of environmental variables were taken from the Basic Halo Measurement and Sensor System (BAHAMAS).

Out of the dates for which all three instruments (BAHAMAS, CAS-DPOL, CCP-CIP) report data, we take those for which measurements of shared variables (true airspeed for BAHAMAS and CAS-DPOL; $\xi$ correction factor for CAS-DPOL and CCP-CIP) are well-correlated ($R^2$ above 0.95). These are (all in 2014): 6, 9, 11, 12, 16, 18, 27, 28, 30 September; 1 October.

We used the same Equation \ref{fullss} for $SS_{QSS}$ and for ventilation factors as described above.

\subsection{CAIPEEX}

Cloud droplet spectra for phase 1 of the CAIPEEX field campaign were measured by a CDP (detection range 2-1562.5 $\mu$m). We used the data from the following flight dates in 2009: 16, 21, 22 June; and 18, 23, 24, 25 August.

We used the same Equation \ref{fullss} for $SS_{QSS}$ and for ventilation factors as described above, and excluded data from cloud droplets of diameter less than 5 $\mu$m.

\begin{figure}[ht]
	\centering
	\begin{subfigure}{0.7\textwidth}
		\includegraphics[width=\textwidth]{revmywrf/v2_FINAL_subdom_bipanel_ss_qss_vs_z_allpts_figure.png}
		\caption{}
		\label{wrfsubdombipanelallpts}
	\end{subfigure}
	\begin{subfigure}{0.7\textwidth}
		\includegraphics[width=\textwidth]{revmywrf/v2_FINAL_subdom_bipanel_ss_qss_vs_z_up10perc_figure.png}
		\caption{}
		\label{wrfsubdombipanelup50perc}
	\end{subfigure}
	\caption{Analagous to Figure \ref{wrfbipanel} restricted to the horizontal subdomain indicated by the red box in Figure S8 (bottom left panel).}
	\label{wrfsubdombipanel}
\end{figure}

\begin{sidewaystable}[]
\centering
\begin{tabular}{@{}llll@{}}
\toprule
Symbol & Meaning & Value of constant & Notes \\ \midrule
$C_{ap}$ & Specific heat capacity at constant pressure, dry air & 1005 J/kg &  \\
$D$ & Molecular diffusion constant of water in dry air & 0.23e-4 m$^2$/s & We take as constant wrt T \\
$e_s$ & Saturation vapor pressure, water & - &  \\
$f(r)$ & Ventilation factor & - &  \\
$g$ & Gravitational acceleration on Earth & 9.8 m/s &  \\
$K$ & Coefficient of thermal conductivity in dry air & 2.4e-2 J/(m s K) & We take as constant wrt T \\
$LWC$ & Liquid water content & - &  \\
$L_v$ & Latent heat of vaporization, water & 2.501e6 J/kg & We take as constant wrt T \\
$N$ & Particle number concentration & - &  \\
$n_{points}$ & Point probability density & - &  \\
$N_{points}$ & Absolute number of points & - &  \\
$q_v$ & Water vapor mass mixing ratio & - & Equals $\frac{m_v}{m_{tot}}$ \\
$q_v^*$ & Saturation water vapor mass mixing ratio & - & Equals $\frac{m_v^*}{m_{tot}}$ \\
$r$ & Particle radius & - &  \\
$RH$ & Relative humidity & - & Equals $SS+1$ \\
$\rho_a$ & Mass density, dry air & - & Assuming ideal gas law \\
$\rho_w$ & Mass density, liquid water & 1000 kg/m$^3$ &  \\
$R_a$ & Ideal gas constant, dry air & 287.19 J/(kg K) &  \\
$R_v$ & Ideal gas constant, water & 460.52 J/(kg K) &  \\
$SS$ & Supersaturation & - & Equals $RH-1$ \\
$T$ & Temperature & - &  \\
$T_c$ & Temperature in degrees C & - & Equals $T – 273.15$ \\
$w$ & Vertical wind velocity & - &  \\
$z$ & Altitude & - &  \\ \bottomrule
\end{tabular}
\caption{Explanation of constants and variables used in the paper.}
\label{vartable}
\end{sidewaystable}

\clearpage
\newpage

\bibliography{refs}
\bibliographystyle{ieeetr}
\end{document}
